\documentclass{minimal}\usepackage[]{graphicx}\usepackage[]{color}
%% maxwidth is the original width if it is less than linewidth
%% otherwise use linewidth (to make sure the graphics do not exceed the margin)
\makeatletter
\def\maxwidth{ %
  \ifdim\Gin@nat@width>\linewidth
    \linewidth
  \else
    \Gin@nat@width
  \fi
}
\makeatother

\definecolor{fgcolor}{rgb}{0.345, 0.345, 0.345}
\newcommand{\hlnum}[1]{\textcolor[rgb]{0.686,0.059,0.569}{#1}}%
\newcommand{\hlstr}[1]{\textcolor[rgb]{0.192,0.494,0.8}{#1}}%
\newcommand{\hlcom}[1]{\textcolor[rgb]{0.678,0.584,0.686}{\textit{#1}}}%
\newcommand{\hlopt}[1]{\textcolor[rgb]{0,0,0}{#1}}%
\newcommand{\hlstd}[1]{\textcolor[rgb]{0.345,0.345,0.345}{#1}}%
\newcommand{\hlkwa}[1]{\textcolor[rgb]{0.161,0.373,0.58}{\textbf{#1}}}%
\newcommand{\hlkwb}[1]{\textcolor[rgb]{0.69,0.353,0.396}{#1}}%
\newcommand{\hlkwc}[1]{\textcolor[rgb]{0.333,0.667,0.333}{#1}}%
\newcommand{\hlkwd}[1]{\textcolor[rgb]{0.737,0.353,0.396}{\textbf{#1}}}%

\usepackage{framed}
\makeatletter
\newenvironment{kframe}{%
 \def\at@end@of@kframe{}%
 \ifinner\ifhmode%
  \def\at@end@of@kframe{\end{minipage}}%
  \begin{minipage}{\columnwidth}%
 \fi\fi%
 \def\FrameCommand##1{\hskip\@totalleftmargin \hskip-\fboxsep
 \colorbox{shadecolor}{##1}\hskip-\fboxsep
     % There is no \\@totalrightmargin, so:
     \hskip-\linewidth \hskip-\@totalleftmargin \hskip\columnwidth}%
 \MakeFramed {\advance\hsize-\width
   \@totalleftmargin\z@ \linewidth\hsize
   \@setminipage}}%
 {\par\unskip\endMakeFramed%
 \at@end@of@kframe}
\makeatother

\definecolor{shadecolor}{rgb}{.97, .97, .97}
\definecolor{messagecolor}{rgb}{0, 0, 0}
\definecolor{warningcolor}{rgb}{1, 0, 1}
\definecolor{errorcolor}{rgb}{1, 0, 0}
\newenvironment{knitrout}{}{} % an empty environment to be redefined in TeX

\usepackage{alltt}
%% from http://www.texample.net/tikz/examples/filesystem-tree/
\usepackage{tikz}
\usetikzlibrary{trees}
\IfFileExists{upquote.sty}{\usepackage{upquote}}{}
\begin{document}
\tikzstyle{every node}=[draw=black,thick,anchor=west]
\tikzstyle{folder}=[fill=gray!30,very thick]
\tikzstyle{rpackage}=[dashed,fill=gray!30,very thick]
\tikzstyle{rpackagefile}=[dashed]
\begin{tikzpicture}[%
  grow via three points={one child at (0.5,-0.7) and
  two children at (0.5,-0.7) and (0.5,-1.4)}, font=\sffamily,
  edge from parent path={(\tikzparentnode.south) |- (\tikzchildnode.west)}]
  \node {Zip file}
    child { node [rpackage] {DESCRIPTION (machine-readable metadata)}}
    child { node [rpackage]  {R (frequently used R functions)}}		
    child { node  [rpackage] {man (documentation for R functions)}}
    child { node  [rpackage] {vignettes}
    child { node [rpackagefile] {R markdown file (main literate programming document)}}
      child { node [folder] {data (CSV files of raw measurement data)}}
      child { node [folder] {figures (generated by the R markdown file)}}
    }
    child [missing] {}				
    child [missing] {}				
    child [missing] {}				
    child { node {Readme file (notes to orient other users)}}
    child { node {Dockerfile (machine-readable instructions to make Docker image)}}
    ;
\end{tikzpicture}
\end{document}
